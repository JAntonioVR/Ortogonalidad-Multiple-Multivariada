\documentclass[11pt,a4paper]{amsart}

\usepackage{latexsym}
\usepackage{amsmath,amsthm,amssymb,amsfonts}
\usepackage{amsmath,amssymb,amsthm,mathrsfs}

\usepackage{verbatim, enumerate}
\usepackage{graphicx}
\usepackage{epstopdf,fancybox,color}
\usepackage{tabularx}
\usepackage{multirow, array}
\usepackage{float}
\usepackage{subcaption}
\usepackage{color}
\usepackage{longtable}
\usepackage[spanish]{babel}
\usepackage[table]{xcolor}


\allowdisplaybreaks

\setlength{\textwidth}{16cm} \setlength{\oddsidemargin}{0.25cm}
\setlength{\evensidemargin}{0.25cm} \setlength{\topmargin}{-1.2cm}
\setlength{\textheight}{24cm}

%\linespread{1.3}


\newcommand{\head}{
{\noindent \small  Seminario GOYA -- 02/04/2024} \vspace{10pt}}
\renewcommand{\refname}{Bibliography}


\newtheorem{theorem}{Theorem}
\newtheorem{remark}{Remark}
\newtheorem{lemma}[theorem]{Lemma}
\newtheorem{proposition}[theorem]{Proposition}
\newtheorem{corollary}[theorem]{Corollary}
\newtheorem{definition}[theorem]{Definition}



\begin{document}

\head \vspace{1cm}


%%%%%%%%%%%%%%%%%%%%%%%%%%%%%%%%%%%%%%%%%%%%%%%%%%%%%%%%%%%%%%%%%%%%%%%%%%%%%%%%%%%%%%%%%%%
%%%%%%%%%%%%%%%%%%%%%%%%%%%%%%%%%%%%%%%%%%%%%%%%%%%%%%%%%%%%%%%%%%%%%%%%%%%%%%%%%%%%%%%%%%%
%               Insert below the tittle, authors, etc.
%               Underline the name of the speaker.
%%%%%%%%%%%%%%%%%%%%%%%%%%%%%%%%%%%%%%%%%%%%%%%%%%%%%%%%%%%%%%%%%%%%%%%%%%%%%%%%%%%%%%%%%%%
%%%%%%%%%%%%%%%%%%%%%%%%%%%%%%%%%%%%%%%%%%%%%%%%%%%%%%%%%%%%%%%%%%%%%%%%%%%%%%%%%%%%%%%%%%%

\title{Avances en la generalización de la ortogonalidad múltiple al caso bivariado.}

\author{\underline{Juan Antonio Villegas}, Lidia Fernández}

\maketitle

%%%%%%%%%%%%%%%%%%%%%%%%%%%%%%%%%%%%%
% Write below the authors and a short tittle for the headlines
%%%%%%%%%%%%%%%%%%%%%%%%%%%%%%%%%%%%%

\markboth{First author, Second author and Last author}{Short title of the talk}

%%%%%%%%%%%%%%%%%%%%%%%%%%%%%%%%%%%%%%%%%%%%%%%%%%%%%%%%%%%%%%%%%%%%%%%%%%%%%%%%%%%%%%%%%%%



\begin{center} \textsc{Resumen}
\end{center}


%%%%%%%%%%%%%%%%%%%%%%%%%%%%%%%%%%%%%%%%%%%%%%%%%%%%%%%%%%%%%%%%%%%%%%%%%%%%%%%%%%%%%%%%%%%
%%%%%%%%%%%%%%%%%%%%%%%%%%%%%%%%%%%%%%%%%%%%%%%%%%%%%%%%%%%%%%%%%%%%%%%%%%%%%%%%%%%%%%%%%%%
%               Insert below your abstract
%%%%%%%%%%%%%%%%%%%%%%%%%%%%%%%%%%%%%%%%%%%%%%%%%%%%%%%%%%%%%%%%%%%%%%%%%%%%%%%%%%%%%%%%%%%
%%%%%%%%%%%%%%%%%%%%%%%%%%%%%%%%%%%%%%%%%%%%%%%%%%%%%%%%%%%%%%%%%%%%%%%%%%%%%%%%%%%%%%%%%%%

La ortogonalidad múltiple es una teoría muy estudiada en los últimos años que extiende el concepto de ortogonalidad estándar en una variable. Los polinomios ortogonales múltiples satisfacen relaciones de ortogonalidad con respecto a más de una medida. Existen dos tipos distintos de ortogonalidad múltiple, la ortogonalidad de tipo I y la de tipo II, que guardan una estrecha relación de equivalencia a pesar de ciertas diferencias. Esta teoría es relevante debido a sus aplicaciones en aproximación racional de Hermite-Padé, matrices aleatorias o sistemas integrables. Sin embargo, no existe aún una extensión de la ortogonalidad múltiple al caso multivariado. En este seminario contaré los avances realizados, los distintos enfoques propuestos y los resultados obtenidos en la investigación de una posible generalización de la ortogonalidad múltiple a polinomios de varias variables.


%%%%%%%%%%%%%%%%%%%%%%%%%%%%%%%%%%%%%%%%%%%%%%%%%%%%%%%%%%%%%%%%%%%%%%%%%%%%%%%%%%%%%%%%%%%


\bigskip
\bigskip

%%%%%%%%%%%%%%%%%%%%%%%%%%%%%%%%%%%%%%%%%%%%%%%%%%%%%%%%%%%%%%%%%%%%%%%%%%%%%%%%%%%%%%%%%%%
%%%%%%%%%%%%%%%%%%%%%%%%%%%%%%%%%%%%%%%%%%%%%%%%%%%%%%%%%%%%%%%%%%%%%%%%%%%%%%%%%%%%%%%%%%%
%               Insert below the keywords and AMS classification
%%%%%%%%%%%%%%%%%%%%%%%%%%%%%%%%%%%%%%%%%%%%%%%%%%%%%%%%%%%%%%%%%%%%%%%%%%%%%%%%%%%%%%%%%%%
%%%%%%%%%%%%%%%%%%%%%%%%%%%%%%%%%%%%%%%%%%%%%%%%%%%%%%%%%%%%%%%%%%%%%%%%%%%%%%%%%%%%%%%%%%%

\noindent
\textit{Palabras clave:} Polinomios Ortogonales, Teoría de Aproximación, Aplicaciones, Ortogonalidad Múltiple.


%%%%%%%%%%%%%%%%%%%%%%%%%%%%%%%%%%%%%%%%%%%%%%%%%%%%%%%%%%%%%%%%%%%%%%%%%%%%%%%%%%%%%%%%%%%





%%%%%%%%%%%%%%%%%%%%%%%%%%%%%%%%%%%%%%%%%%%%%%%%%%%%%%%%%%%%%%%%%%%%%%%%%%%%%%%%%%%%%%%%%%%
%%%%%%%%%%%%%%%%%%%%%%%%%%%%%%%%%%%%%%%%%%%%%%%%%%%%%%%%%%%%%%%%%%%%%%%%%%%%%%%%%%%%%%%%%%%
%               Insert the Bibliography
%               Use a format similar to the examples below
%%%%%%%%%%%%%%%%%%%%%%%%%%%%%%%%%%%%%%%%%%%%%%%%%%%%%%%%%%%%%%%%%%%%%%%%%%%%%%%%%%%%%%%%%%%
%%%%%%%%%%%%%%%%%%%%%%%%%%%%%%%%%%%%%%%%%%%%%%%%%%%%%%%%%%%%%%%%%%%%%%%%%%%%%%%%%%%%%%%%%%%

\begin{thebibliography}{99}

\bibitem{ismail}
     M. E. H. Ismail,
     \emph{Classical and quantum orthogonal polynomials in one variable}, Encyclopedia of mathematics and its applications, Cambridge University Press (2005). 
\bibitem{dunkl_xu_2014}
     C. F. Dunkl and Y. Xu,
     \emph{Orthogonal Polynomials of Several Variables}, Cambridge University Press (2014).

\bibitem{andrei}
     A. Martínez-Finkelshtein and W. van Assche,
     \emph{What is \dots a multiple orthogonal polynomial?}, Notices of the American Mathematical Society, 63 (2016), 1029--1031.

\bibitem{walter}
     W. Van Assche
     \emph{Orthogonal and multiple orthogonal polynomials, random matrices, and painlevé equations}, ``Orthogonal Polynomials'' (M. Foupouagnigni, W. Koepf, eds), Tutorials, Schools and Workshops in the Mathematical Sciences, Springer Nature Switzerland (2020) 629--683.
     
\end{thebibliography}


%%%%%%%%%%%%%%%%%%%%%%%%%%%%%%%%%%%%%%%%%%%%%%%%%%%%%%%%%%%%%%%%%%%%%%%%%%%%%%%%%%%%%%%%%%%


\bigskip


%%%%%%%%%%%%%%%%%%%%%%%%%%%%%%%%%%%%%%%%%%%%%%%%%%%%%%%%%%%%%%%%%%%%%%%%%%%%%%%%%%%%%%%%%%%
%%%%%%%%%%%%%%%%%%%%%%%%%%%%%%%%%%%%%%%%%%%%%%%%%%%%%%%%%%%%%%%%%%%%%%%%%%%%%%%%%%%%%%%%%%%
%               Insert here the complete addresses of the authors
%%%%%%%%%%%%%%%%%%%%%%%%%%%%%%%%%%%%%%%%%%%%%%%%%%%%%%%%%%%%%%%%%%%%%%%%%%%%%%%%%%%%%%%%%%%
%%%%%%%%%%%%%%%%%%%%%%%%%%%%%%%%%%%%%%%%%%%%%%%%%%%%%%%%%%%%%%%%%%%%%%%%%%%%%%%%%%%%%%%%%%%

\mbox{ }


\end{document}
