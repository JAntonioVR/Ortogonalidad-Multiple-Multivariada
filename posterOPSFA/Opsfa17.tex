\documentclass[11pt,a4paper]{amsart}

\usepackage{latexsym}
\usepackage{amsmath,amsthm,amssymb,amsfonts}
\usepackage{amsmath,amssymb,amsthm,mathrsfs}

\usepackage{verbatim, enumerate}
\usepackage{graphicx}
\usepackage{epstopdf,fancybox,color}
\usepackage{tabularx}
\usepackage{multirow, array}
\usepackage{float}
\usepackage{subcaption}
\usepackage{color}
\usepackage{longtable}
\usepackage[table]{xcolor}


\allowdisplaybreaks

\setlength{\textwidth}{16cm} \setlength{\oddsidemargin}{0.25cm}
\setlength{\evensidemargin}{0.25cm} \setlength{\topmargin}{-1.2cm}
\setlength{\textheight}{24cm}

%\linespread{1.3}


\newcommand{\head}{
{\noindent \small IMAG - OPSFA17.
Granada (Spain) June 24 - 28, 2024} \vspace{10pt}}
\renewcommand{\refname}{Bibliography}


\newtheorem{theorem}{Theorem}
\newtheorem{remark}{Remark}
\newtheorem{lemma}[theorem]{Lemma}
\newtheorem{proposition}[theorem]{Proposition}
\newtheorem{corollary}[theorem]{Corollary}
\newtheorem{definition}[theorem]{Definition}



\begin{document}

\head \vspace{1cm}


%%%%%%%%%%%%%%%%%%%%%%%%%%%%%%%%%%%%%%%%%%%%%%%%%%%%%%%%%%%%%%%%%%%%%%%%%%%%%%%%%%%%%%%%%%%
%%%%%%%%%%%%%%%%%%%%%%%%%%%%%%%%%%%%%%%%%%%%%%%%%%%%%%%%%%%%%%%%%%%%%%%%%%%%%%%%%%%%%%%%%%%
%               Insert below the tittle, authors, etc.
%               Underline the name of the speaker.
%%%%%%%%%%%%%%%%%%%%%%%%%%%%%%%%%%%%%%%%%%%%%%%%%%%%%%%%%%%%%%%%%%%%%%%%%%%%%%%%%%%%%%%%%%%
%%%%%%%%%%%%%%%%%%%%%%%%%%%%%%%%%%%%%%%%%%%%%%%%%%%%%%%%%%%%%%%%%%%%%%%%%%%%%%%%%%%%%%%%%%%

\title{Generalization of the multiple orthogonality to the bivariate case}

\author{\underline{J. Antonio Villegas}, Lidia Fernández}

\thanks{Universidad de Granada; IMAG-María de Maeztu, grant ``IMAG CEX2020–001105–M''; GOYA: ``Grupo de Ortogonalidad Y Aplicaciones'' and ``Departamento de Matemática Aplicada''}

\maketitle

%%%%%%%%%%%%%%%%%%%%%%%%%%%%%%%%%%%%%
% Write below the authors and a short tittle for the headlines
%%%%%%%%%%%%%%%%%%%%%%%%%%%%%%%%%%%%%

\markboth{First author, Second author and Last author}{Short title of the talk}

%%%%%%%%%%%%%%%%%%%%%%%%%%%%%%%%%%%%%%%%%%%%%%%%%%%%%%%%%%%%%%%%%%%%%%%%%%%%%%%%%%%%%%%%%%%



\begin{center} \textsc{Abstract}
\end{center}


%%%%%%%%%%%%%%%%%%%%%%%%%%%%%%%%%%%%%%%%%%%%%%%%%%%%%%%%%%%%%%%%%%%%%%%%%%%%%%%%%%%%%%%%%%%
%%%%%%%%%%%%%%%%%%%%%%%%%%%%%%%%%%%%%%%%%%%%%%%%%%%%%%%%%%%%%%%%%%%%%%%%%%%%%%%%%%%%%%%%%%%
%               Insert below your abstract
%%%%%%%%%%%%%%%%%%%%%%%%%%%%%%%%%%%%%%%%%%%%%%%%%%%%%%%%%%%%%%%%%%%%%%%%%%%%%%%%%%%%%%%%%%%
%%%%%%%%%%%%%%%%%%%%%%%%%%%%%%%%%%%%%%%%%%%%%%%%%%%%%%%%%%%%%%%%%%%%%%%%%%%%%%%%%%%%%%%%%%%

Polynomials known as Multiple Orthogonal Polynomials (MOPs) in a single variable are polynomials that satisfy orthogonality conditions concerning multiple measures and play significant role in several applications such as Hermite-Padé approximation, random matrix theory or integrable systems. However, this theory has only been studied in the univariate case. In this poster, some generalized definitions of the two main types of multiple orthogonality are given, together with some examples and extended results. 


%%%%%%%%%%%%%%%%%%%%%%%%%%%%%%%%%%%%%%%%%%%%%%%%%%%%%%%%%%%%%%%%%%%%%%%%%%%%%%%%%%%%%%%%%%%


\bigskip
\bigskip

%%%%%%%%%%%%%%%%%%%%%%%%%%%%%%%%%%%%%%%%%%%%%%%%%%%%%%%%%%%%%%%%%%%%%%%%%%%%%%%%%%%%%%%%%%%
%%%%%%%%%%%%%%%%%%%%%%%%%%%%%%%%%%%%%%%%%%%%%%%%%%%%%%%%%%%%%%%%%%%%%%%%%%%%%%%%%%%%%%%%%%%
%               Insert below the keywords and AMS classification
%%%%%%%%%%%%%%%%%%%%%%%%%%%%%%%%%%%%%%%%%%%%%%%%%%%%%%%%%%%%%%%%%%%%%%%%%%%%%%%%%%%%%%%%%%%
%%%%%%%%%%%%%%%%%%%%%%%%%%%%%%%%%%%%%%%%%%%%%%%%%%%%%%%%%%%%%%%%%%%%%%%%%%%%%%%%%%%%%%%%%%%

\noindent
\textit{Keywords:} Orthogonal Polynomials, Approximation Theory, Applications, Multiple orthogonality.


\noindent
\textit{AMS Classification:}  33C45, 33C50, 42C05.


%%%%%%%%%%%%%%%%%%%%%%%%%%%%%%%%%%%%%%%%%%%%%%%%%%%%%%%%%%%%%%%%%%%%%%%%%%%%%%%%%%%%%%%%%%%





%%%%%%%%%%%%%%%%%%%%%%%%%%%%%%%%%%%%%%%%%%%%%%%%%%%%%%%%%%%%%%%%%%%%%%%%%%%%%%%%%%%%%%%%%%%
%%%%%%%%%%%%%%%%%%%%%%%%%%%%%%%%%%%%%%%%%%%%%%%%%%%%%%%%%%%%%%%%%%%%%%%%%%%%%%%%%%%%%%%%%%%
%               Insert the Bibliography
%               Use a format similar to the examples below
%%%%%%%%%%%%%%%%%%%%%%%%%%%%%%%%%%%%%%%%%%%%%%%%%%%%%%%%%%%%%%%%%%%%%%%%%%%%%%%%%%%%%%%%%%%
%%%%%%%%%%%%%%%%%%%%%%%%%%%%%%%%%%%%%%%%%%%%%%%%%%%%%%%%%%%%%%%%%%%%%%%%%%%%%%%%%%%%%%%%%%%

\begin{thebibliography}{99}

\bibitem{ismail}
     M. E. H. Ismail,
     \emph{Classical and quantum orthogonal polynomials in one variable}, Encyclopedia of mathematics and its applications, Cambridge University Press (2005). 
\bibitem{dunkl_xu_2014}
     C. F. Dunkl and Y. Xu
     \emph{Orthogonal Polynomials of Several Variables}, Cambridge University Press (2014).


\bibitem{walter}
     W. Van Assche
     \emph{Orthogonal and multiple orthogonal polynomials, random matrices, and painlevé equations}, ``Orthogonal Polynomials'' (M. Foupouagnigni, W. Koepf, eds), Tutorials, Schools and Workshops in the Mathematical Sciences, Springer Nature Switzerland (2020) 629--683.
     
\end{thebibliography}


%%%%%%%%%%%%%%%%%%%%%%%%%%%%%%%%%%%%%%%%%%%%%%%%%%%%%%%%%%%%%%%%%%%%%%%%%%%%%%%%%%%%%%%%%%%


\bigskip


%%%%%%%%%%%%%%%%%%%%%%%%%%%%%%%%%%%%%%%%%%%%%%%%%%%%%%%%%%%%%%%%%%%%%%%%%%%%%%%%%%%%%%%%%%%
%%%%%%%%%%%%%%%%%%%%%%%%%%%%%%%%%%%%%%%%%%%%%%%%%%%%%%%%%%%%%%%%%%%%%%%%%%%%%%%%%%%%%%%%%%%
%               Insert here the complete addresses of the authors
%%%%%%%%%%%%%%%%%%%%%%%%%%%%%%%%%%%%%%%%%%%%%%%%%%%%%%%%%%%%%%%%%%%%%%%%%%%%%%%%%%%%%%%%%%%
%%%%%%%%%%%%%%%%%%%%%%%%%%%%%%%%%%%%%%%%%%%%%%%%%%%%%%%%%%%%%%%%%%%%%%%%%%%%%%%%%%%%%%%%%%%
{\footnotesize\noindent J. Antonio Villegas,\\
Instituto de Matemáticas (IMAG) and Departamento de Matemática Aplicada,\\
Universidad de Granada.\\
{\tt jantoniovr@ugr.es}

\bigskip

\noindent Lidia Fernández,\\
Instituto de Matemáticas (IMAG) and Departamento de Matemática Aplicada,\\
Universidad de Granada.\\
{\tt lidiafr@ugr.es}

\bigskip


\mbox{ }}


\end{document}
